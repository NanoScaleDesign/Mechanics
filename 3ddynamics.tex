\newpage
\section{Radial velocity with horizontal connection}

\subsection*{Resources}
\begin{itemize}
    \item Book sections 7/1 to 7/5
\end{itemize}

\emph{Correction to book: Figure 7/9 should read $\bm{\alpha} = \bm{\dot{\omega}} = \bm{\Omega} \bm{\times} \bm{\omega}$ (not $\bm{\Omega} \bm{\times} \bm{r}$)}

\subsection*{Challenge}
A weight ``A'' is tethered to a pole by a stiff rod of length $r$. If the angular velocity is \SI{5.5}{\radian\per\second} $\hat{k}$ and the length of the rod is \SI{47}{\meter} along the x-axis, what is the linear velocity of the weight ``A''?

\includegraphics[height=5cm]{rod-horizontal}

\subsection*{Solution}
X = Your solution\\
Units: \si{\meter\per\second}\\
Form: Decimal, to 1 decimal place\\
Place the indicated letter in front of the number\\
Example: aX where $X=42.5$ \si{\meter\per\second} is entered as \href{http://www.wolframalpha.com/input/?i=md5+hash+of+\%22a42.5\%22}{a42.5}

$\hat{i}=$ hash of aX = 9497cd \si{\meter\per\second}\\
$\hat{j}=$ hash of bX = d17e5c \si{\meter\per\second}\\
$\hat{k}=$ hash of cX = 347133 \si{\meter\per\second}




\newpage
\section{Radial velocity with non-horizontal connection}

\subsection*{Resources}
\begin{itemize}
    \item Book sections 7/1 to 7/5
\end{itemize}

\emph{Correction to book: Figure 7/9 should read $\bm{\alpha} = \bm{\dot{\omega}} = \bm{\Omega} \bm{\times} \bm{\omega}$ (not $\bm{\Omega} \bm{\times} \bm{r}$)}

\subsection*{Challenge}
1. The position of ``A'' and the pole are unchanged (the radial distance is the same) and the angular velocity remains the same, but ``A'' is now hinged to the pole from below instead of horizontally, as shown in the picture. Calculate the linear velocity of ``A'' (calculate mathematically, not just by comparison with the previous challenge).

2. Write a sentence or two comparing your answer with that obtained from the previous challenge, including reasoning why.

\includegraphics[height=5cm]{rod-frombelow}

\subsection*{Solution}
1.\\
X = Your solution\\
Units: \si{\meter\per\second}\\
Form: Decimal, to 1 decimal place\\
Place the indicated letter in front of the number\\
Example: aX where $X=42.5$ \si{\meter\per\second} is entered as \href{http://www.wolframalpha.com/input/?i=md5+hash+of+\%22a42.5\%22}{a42.5}

$\hat{i}=$ hash of dX = c6e675 \si{\meter\per\second}\\
$\hat{j}=$ hash of eX = bcff19 \si{\meter\per\second}\\
$\hat{k}=$ hash of fX = 979ed8 \si{\meter\per\second}

2. Please discuss in class if you are unsure about your answer.




\newpage
\section{Linear acceleration}

\subsection*{Resources}
\begin{itemize}
    \item Book sections 7/1 to 7/5
\end{itemize}

\emph{Correction to book: Figure 7/9 should read $\bm{\alpha} = \bm{\dot{\omega}} = \bm{\Omega} \bm{\times} \bm{\omega}$ (not $\bm{\Omega} \bm{\times} \bm{r}$)}

\subsection*{Challenge}
Using information from previous challenges, determine the:

1. Linear acceleration towards the centre of pole.

2. The tangential linear acceleration

3. Is there linear acceleration towards the centre of the pole? Is there tangential linear acceleration? Write a sentence or two to explain why for both cases.

\includegraphics[height=5cm]{rod-frombelow}


\subsection*{Solution}
X = Your solution\\
Units: \si{\meter\per\square\second}\\
Form: Decimal, to 2 decimal place\\
Place the indicated letter in front of the number\\
Example: aX where $X=42.57$ \si{\meter\per\second} is entered as \href{http://www.wolframalpha.com/input/?i=md5+hash+of+\%22a42.5\%22}{a42.57}

1.\\
$\hat{i}=$ hash of gX = e1993f \si{\meter\per\square\second}\\
$\hat{j}=$ hash of hX = 5a16a7 \si{\meter\per\square\second}\\
$\hat{k}=$ hash of iX = 2ebd7c \si{\meter\per\square\second}

2.\\
$\hat{i}=$ hash of jX = 4b3090 \si{\meter\per\square\second}\\
$\hat{j}=$ hash of kX = 28435f \si{\meter\per\square\second}\\
$\hat{k}=$ hash of lX = 060ec3 \si{\meter\per\square\second}

3. Please compare your answer with your partner.




\newpage
\section{Radial acceleration - only magnitude}

\subsection*{Resources}
\begin{itemize}
    \item Book sections 7/1 to 7/5
\end{itemize}

\emph{Correction to book: Figure 7/9 should read $\bm{\alpha} = \bm{\dot{\omega}} = \bm{\Omega} \bm{\times} \bm{\omega}$ (not $\bm{\Omega} \bm{\times} \bm{r}$)}

\subsection*{Challenge}
Now consider that the radial velocity is not constant, but is undergoing an acceleration so that the magnitude of the angular velocity $\bm{w}$ increases while it continues to point in the same direction.

If the acceleration is \SI{2}{\radian\per\square\second}, what is the tangential acceleration of ``A''?

\includegraphics[height=5cm]{rod-frombelow}

\subsection*{Solution}
X = Your solution\\
Units: \si{\meter\per\second}\\
Form: Decimal, to 1 decimal place\\
Place the indicated letter in front of the number\\
Example: aX where $X=42.5$ \si{\meter\per\second} is entered as \href{http://www.wolframalpha.com/input/?i=md5+hash+of+\%22a42.5\%22}{a42.5}

$\hat{i}=$ hash of mX = b9f8f5 \si{\meter\per\second}\\
$\hat{j}=$ hash of nX = 57e394 \si{\meter\per\second}\\
$\hat{k}=$ hash of oX = 0c8b72 \si{\meter\per\second}





\newpage
\section{Radial acceleration - only direction (precession)}

\subsection*{Resources}
\begin{itemize}
    \item Book sections 7/1 to 7/5
\end{itemize}

\emph{Correction to book: Figure 7/9 should read $\bm{\alpha} = \bm{\dot{\omega}} = \bm{\Omega} \bm{\times} \bm{\omega}$ (not $\bm{\Omega} \bm{\times} \bm{r}$)}

\subsection*{Challenge}
The previous challenges considered the case (a) below, where the direction of the angular velocity vector $\omega$ was unchanging. Next consider that the angular velocity vector is precessing around an axis of symettry, and this precession has an angular velocity of $\Omega$, as shown in (b). Combining (a) and (b) we have (c).

\includegraphics[height=5cm]{precession}

1. Assuming that only the direction (not the magnitude) of the angular velocity vector $\omega$ is changing with time, calculate the acceleration of ``A'' if $\Omega = 3\hat{k}$ \si{\radian\per\second} and angular velocity vector $\omega$ is inclined at \SI{45}{\degree} with components $\omega = 5.5 \hat{i} + 5.5 \hat{k}$.

2. What is the direction of the acceleration of the angular velocity vector $\omega$. Write 1 or 2 sentences to explain why.

3. What is the direction of the acceleration of ``A''. Write one or two sentences (possibly with a diagram) to explain when the sign will be opposite but with same magnitude.


\subsection*{Solution}
1.\\
X = Your solution\\
Units: \si{\meter\per\second}\\
Form: Decimal, to 1 decimal place\\
Place the indicated letter in front of the number\\
Example: aX where $X=42.5$ \si{\meter\per\second} is entered as \href{http://www.wolframalpha.com/input/?i=md5+hash+of+\%22a42.5\%22}{a42.5}

$\hat{i}=$ hash of pX = 8829fa \si{\meter\per\second}\\
$\hat{j}=$ hash of qX = c573be \si{\meter\per\second}\\
$\hat{k}=$ hash of rX = 6f484b \si{\meter\per\second}

2. and 3. Please discuss in class.




\newpage
\section{Radial acceleration II}

\subsection*{Resources}
\begin{itemize}
    \item Book sections 7/1 to 7/5
\end{itemize}

\emph{Correction to book: Figure 7/9 should read $\bm{\alpha} = \bm{\dot{\omega}} = \bm{\Omega} \bm{\times} \bm{\omega}$ (not $\bm{\Omega} \bm{\times} \bm{r}$)}

\subsection*{Challenge}
Question 7/4

\subsection*{Solution}
\SI{1285}{\meter\per\second}
