\newpage
\section{Coursework}
\course is a large subject with a wide-range of applications. This coursework is designed to give you the opportunity to follow your personal interest and investigate in depth an area of \course of your choice.

The task is as follows:

\textbf{1)} Create a document, explaining about any application of \course that interests you. The document should be \textbf{between 1 and 4 pages}, including any necessary figures, mathematics and references.

\textbf{2)} Create \textbf{1 to 2 challenge(s)} to accompany your report, so someone reading your document can test their knowledge.

\textbf{3)} Include \textbf{fully worked} solutions to challenges you make (ie, not only the final answer, but clearly show the steps involved in order to achieve the final answer).

\subsection{Submission}
Submission is electronic, and the file may be in any format, including PDF, LibreOffice, MS Word, Google docs, Latex, etc\ldots If you submit a PDF, please also submit the source-files used to generate the PDF.

Submit the materials by \textbf{email} to the teacher \textbf{before the class on 10 January 2017} with the subject ``[\coursenospace] Coursework''. I will confirm in the class that I received your coursework. If you cannot attend the class, you must request confirmation of receipt when you send the email.

Late submission:\\
By 23:59 on 11 July 2017: 90\% of the final mark.\\
By 23:59 on 17 July 2017: 50\% of the final mark.\\
Later submissions cannot be considered.

\subsection{Marking}
Marks will be assigned based on the degree to the report fulfills the following criteria:
For maximum marks you should do the following:
\begin{itemize}
    \item Clearly demonstrate your understanding of what you write about. You can do this by, for example, mathematically solving for a relevant case or explaining with words how it applies in different situations.
    \item Ensure your subject has some relation to Mechanics and is of Engineering relevance.
    \item Choose a subject that goes beyond the boundaries of the examples covered in the textbook.
    \item Ensure the work is your own and all references, images and text taken from other sources are properly cited.
    \item Pitch the description at a level appropriate level so that others in the class can follow your discussion.
    \item Explain in reasonable depth.
    \item Explain accurately and clearly.
\end{itemize}

\emph{Note: The application that you describe can does not have to be originally invented by you (although you are welcome to propose an application like this if you wish). The application may already exist, but you will need to demonstrate understanding about the application and calculations involved in the use of \course with this application.}
