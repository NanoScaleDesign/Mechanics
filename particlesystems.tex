\newpage
\section{System centre-of-mass position, mass and velocity: I}

\subsection*{Resources}
\begin{itemize}
    \item Book sections 4/1 to 4/2
\end{itemize}

\subsection*{Challenge}
1. $\bm{\com{r}}$, $\dot{\bm{\com{r}}}$ and $\ddot{\bm{\com{r}}}$ of Question 4/1.

2. Question 4/4, but replace the word ``lb'' with ``kg'' and the word ``ft'' with ``m''. \emph{Do not convert units mathematically; just replace word-for-word. I realise that this will make for heavy monkeys.}

\subsection*{Solution}
1. Given in book.

2. \SI{533.6}{N}




\newpage
\section{System centre-of-mass position, mass and velocity: II}

\subsection*{Resources}
\begin{itemize}
    \item Book sections 4/1 to 4/2
\end{itemize}

\subsection*{Challenge}
Question 4/5. Determine the \emph{magnitude} of the acceleration.

\subsection*{Solution}
Given in book.




\newpage
\section{System centre-of-mass position, mass and velocity: III}

\subsection*{Resources}
\begin{itemize}
    \item Book sections 4/1 to 4/2
\end{itemize}

\subsection*{Challenge}
Question 4/13, but change the force to a \SI{10}{N} force and the mass of each bar to \SI{8}{kg}.

\subsection*{Solution}
\SI{0.42}{m/s^2}




\newpage
\section{Kinetic and potential energy}

\subsection*{Resources}
\begin{itemize}
    \item Book section 4/3
\end{itemize}

\subsection*{Challenge}
1. Calculate $T$ in question 4/1

2. Question 4/10

\subsection*{Solution}
1. Given in book.

2. To check your final answer, substitute b = 2 metres into your final answer. You should obtain \SI{5.27}{m/s}.




\newpage
\section{Cross-product}

\subsection*{Resources}
\begin{itemize}
    \item \url{https://www.khanacademy.org/science/physics/magnetic-forces-and-magnetic-fields/electric-motors/v/calculating-dot-and-cross-products-with-unit-vector-notation}
\end{itemize}


\subsection*{Challenge}
1. Determine the angle between the two vectors $\bm{a} = [3,0,0]$ and $\bm{b} = [3,1,0]$ and use it to calculate $\bm{c} = \bm{a} \times \bm{b}$. Which direction does the vector $\bm{c}$ point?

2. Determine the cross product $\bm{f} = \bm{d} \times \bm{e}$ where $\bm{d} = 4 \hat{i}+ 2 \hat{j} + 1 \hat{k}$ and $\bm{e} = -2 \hat{i} -4 \hat{j} + 8 \hat{k}$ without calculating the angle between them.

\subsection*{Solution}
Please compare your answer with your partner and discuss in class if answers differ.




\newpage
\section{Rotation I}

\subsection*{Resources}
\begin{itemize}
    \item Book section 4/4
\end{itemize}

\subsection*{Challenges}
Calculate the angular momentum and the rate of change of angular momentum with time for Question 4/1.

\subsection*{Solutions}
Given in book.




\newpage
\section{Rotation II}

\subsection*{Resources}
\begin{itemize}
    \item Book section 4/4
\end{itemize}

\subsection*{Challenges}
Question 4/16

If you have difficulty, consider doing question 4/15 first (optional).

\subsection*{Solutions}
The required time should be \SI{2.72}{s}




\newpage
\section{Rotation III}

\subsection*{Resources}
\begin{itemize}
    \item Book section 4/4
\end{itemize}

\subsection*{Challenges}
Question 4/2

\subsection*{Solutions}
To check your answers substitute $d=2$ metres, $m=7$ kg, $v=3$ m/s and $f=7$ N into your final answers.
You should obtain
$\bm{H}_G = 432 \hat{i} + 144 \hat{j} + 168 \hat{k}$ \SI{}{kg m^2 /s}
and
$\dot{\bm{H}}_G = -8 \hat{i} - 12 \hat{j} + 0 \hat{k}$ \SI{}{Nm}




\newpage
\section{Conservation of momentum}

\subsection*{Resources}
\begin{itemize}
    \item Book section 4/5
\end{itemize}

\subsection*{Challenges}
1. In Question 4/17, at what point does the vehicle stop accelerating? % NT: Change this to calculating velocity first. Add a question about conservation of KE and PE before as well

2. Solve Question 4/17

3. Question 4/18

\subsection*{Solutions}
1. Please write your answer and compare with your partner in class

2. Given in book

3. 0.21 m/s




\newpage
\section{Conservation of momentum vs energy}

\subsection*{Resources}
\begin{itemize}
    \item Book section 4/5
\end{itemize}

\subsection*{Challenges}
1. Solve Question 4/19

2. Why is energy not conserved here? Where did the energy go? Under what conditions is momentum conserved, and under what conditions is energy conserved?

\subsection*{Solutions}
1. Given in book

2. Please write your answers and compare with your partner in class.




\newpage
\section{Combined problems I}

\subsection*{Resources}
\begin{itemize}
    \item Book section 4/1 to 4/5
\end{itemize}

\subsection*{Challenge}
Solve Question 4/22.

The question states that an impulse is imparted ``over a negligibly short period of time'' which is a little confusing since impulse is the integration of force over time which becomes zero as time goes to zero. Instead, here you can consider that whatever the time is, the final product of Force and Time is \SI{10}{\newton\second}.

\subsection*{Solution}
\SI{4.7}{m/s}




\newpage
\section{Combined problems II}

\subsection*{Resources}
\begin{itemize}
    \item Book section 4/1 to 4/5
\end{itemize}

\subsection*{Challenge}
Solve Question 4/28

\subsection*{Solutions}
You should obtain an algebraic expression for $v$ and $\dot{\theta}$. To check your expression, you can substitute the following values into the expression:
$m_0=$ \SI{1}{\kg},
$v_0=$ \SI{1000}{\meter\per\second},
$b=$ \SI{1.5}{\meter} and
$m=$ \SI{4}{\kg}, whereby you should obtain
$v=$ \SI{111}{\meter\per\second} and
$\dot{\theta}=$ \SI{222}{\radian\per\second}.




\newpage
\section{In-plane flow}

\subsection*{Resources}
\begin{itemize}
    \item Book section 4/6
\end{itemize}

\subsection*{Challenge}
Derive equation 4/19 in the book from equation 4/19a.




\newpage
\section{Force on vane}

\subsection*{Resources}
\begin{itemize}
    \item Book section 4/6
\end{itemize}

\subsection*{Challenge}
Show your working for sample problem 4/5 (a) and (b)




\newpage
\section{Power and a vane}

\subsection*{Resources}
\begin{itemize}
    \item Book section 4/6
\end{itemize}

\subsection*{Challenge}
Considering sample problem 4/6,

1. Explain in words what is meant by ``power by action of the fluid''.

2. The power is defined here by measuring the force applied to move an object at a constant velocity. If force creates acceleration ($F=ma$), how can the velocity be constant?

3. The power is zero for either $F=0$ or $u=0$ (arbitrary units). Explain using words the conditions under which these would occur and why this results in no power.

4. Work through and solve the sample problem.

\subsection*{Solutions}
Please compare your solutions with your partner. You may be asked to present your solutions to the class.




\newpage
\section{Balancing forces: Jet aeroplane example}

\subsection*{Resources}
\begin{itemize}
    \item Book section 4/6
\end{itemize}

\subsection*{Challenge}
Work through sample problem 4/8 to obtain the equation of motion of the system.




\newpage
\section{Balancing forces: Jet aeroplane}

\subsection*{Resources}
\begin{itemize}
    \item Book section 4/6
\end{itemize}

\subsection*{Challenge}
Answer question 4/33.

\subsection*{Solution}
Given in book.



\newpage
\section{Balancing forces: Fire tug}

\subsection*{Resources}
\begin{itemize}
    \item Book section 4/6
\end{itemize}

\subsection*{Challenge}
Answer question 4/37.

\subsection*{Solution}
Given in book.




\iffalse
\newpage
\section{Balancing ball on a water stream}

\subsection*{Resources}
\begin{itemize}
    \item Book section 4/6
\end{itemize}

\subsection*{Challenge}
Answer question 4/42

\subsection*{Solution}
\SI{4.8}{\meter}




\newpage
\section{Pressure I}

\subsection*{Resources}
\begin{itemize}
    \item Book section 4/6
\end{itemize}

\subsection*{Challenge}
Consider a dice with side length \SI{1.4}{\cm} weighing \SI{2.8}{\gram} sitting on a desk. Estimate the pressure on the bottom of the dice.

\subsection*{Solution}
\SI{140}{\pascal}




\newpage
\section{Pressure II}

\subsection*{Resources}
\begin{itemize}
    \item Book section 4/6
\end{itemize}

\subsection*{Challenge}
Answer question 4/50

\subsection*{Solution}
\SI{1035}{\pascal}




%\newpage
%\section{Hose}
%
%\subsection*{Resources}
%\begin{itemize}
    %\item Book section 4/6
%\end{itemize}
%
%\subsection*{Challenge}
%Consider question 4/62. Note that the volume flow rate $Q$, measured in $m^3/s$, is the total system volume flow rate, and therefore the volume flow rate emerging from each nozzel is one quater of this for each nozzel.
%
%1. Write an expression for the velocity parallel and perpendicular to rotation for the case where $b$ tends to zero.
%
%2. Write an expression for the velocity parallel and perpendicular to rotation for the case where $r$ tends to zero.
%
%3. Remembering that components of angular velocity can be combined linearly independently, write an expression for the velocity parallel and perpendicular to rotation for the case where neither $b$ or $r$ are tending to zero.
%
%Answer question 4/62
%
%\subsection*{Solution}



\newpage
\section{Helicopter}

\subsection*{Resources}
\begin{itemize}
    \item Book section 4/6
\end{itemize}

\subsection*{Challenge}
Answer question 4/59

\subsection*{Solutions}
Given in book
\fi
